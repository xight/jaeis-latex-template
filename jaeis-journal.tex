\documentclass[dvipdfmx]{jaeis-journal}

\usepackage{graphicx}
\usepackage{amssymb} % 記号

% 表関連
\usepackage{colortbl,array,xcolor}
\usepackage{tabularx}
\newcolumntype{C}{>{\centering}X}
\renewcommand{\tabularxcolumn}[1]{m{#1}}
\usepackage{booktabs}

% 箇条書き
\usepackage{enumitem}

% citeを上付き (X) に変更
\usepackage{cite}

\let\cite=\citen
\renewcommand\citeform[1]{\hspace{-0.1mm}\textsuperscript{(#1)}\hspace{-1mm}}

% 参考文献のリストを (X) に変更
\makeatletter 
\renewcommand{\@biblabel}[1]{(#1)} 
\makeatother

\title{日本情報科教育学会誌}
%\etitle{How to Write Your Paper for the JAEIS Research Reports}
\subtitle{原稿執筆要領}
%\esubtitle{Towards Open Access}
\abstract{ここには,研究の概要をお書きください.400字以内で,目的,方法,結果を記述し,最後の行に,キーワード(3語以上6語以内)を明記してください.提出原稿は,鮮明で読みやすく,正確な出版物とするために,以下の執筆要領にしたがい,原稿の作成をお願いいたします.また,執筆要領で指定されているスタイルにそっていない原稿は,修正をお願いすることがあります.}
\keyword{
	情報科教育,高校教育,教育効果,キーワード4,キーワード5
}

\author{情報 太郎\hspace{3cm}情報 花子\hspace{3cm}情報 次郎}
%\eauthor{Taro Joho$^\ast$\hspace{1cm}Hanako Joho$^\ast$\hspace{1cm}Jiro Joho$^{\ast\ast}$}

\affiliation{情報大学教育学部\hspace{2cm}情報大学工学部\hspace{2cm}情報大学大学院大学}
%\eaffiliation{
%Faculty of Engineering, Joho University$^\ast$\\
%Faculty of Education, Joho University$^{\ast\ast}$}

\email{taro\_joho@xxx.yyy.ac.jp\hspace{5mm}hanako\_joho@zzz.yyy.ac.jp\hspace{5mm}jiro\_joho@kkk.lll.ac.jp}

% 下線
\usepackage{ulinej}

% レイアウトの確認
\usepackage{layout}

\begin{document}

%\layout

\maketitle

\section{はじめに}

\ulinej{論文を投稿する際には,まず,日本情報科教育学会論文投稿要領をご確認ください.}

このWordファイルには,論文タイトル,章,節,参考文献などそれぞれの書式を例示していますので,参考にしてください.

原稿の提出期限は,学会のWebをご覧ください.また,原稿のファイル形式は,Word形式(docx形式)でお願いします.図や本文など,レイアウトの確認のため,DOCファイルと共にPDFファイルも合わせて送信願います.

\section{原稿の作成について}
\subsection{原稿サイズとページ数}

原稿サイズはA4です.印刷される学会誌もA4サイズとなります.原則,頁数は最大10頁までとします.なお,原稿には頁番号を記入しないでください.

\subsection{原稿の余白}

原稿の余白は,上端20mm,下端24mm,左右23mmにしてください.その中の範囲を原稿記入範囲とさせていただきます.

\subsection{使用言語}

原稿に使用する言語は,日本語または英語でお願いします.

\subsection{頁構成}

先頭頁の原稿記入範囲の上部より順に,論文タイトル,著者名,所属の各項目を,1段組でセンタリングでして記入してください.このサンプルのように,副タイトルをつけても構いません.その後,研究概要,キーワードも記入してください.

本文は,2段組で作成してください.1頁あたりの文字数は24文字,行数は45行としてください.

\subsection{文字のフォントとサイズ}

フォントは,MS明朝,MSゴシックを用い,特殊なフォントの使用は避けてください.文字サイズ(ポイント)は,学会誌全体でのバランスを取るために,表1を参考にお願いします.タイトルや章見出し等,このファイルの「スタイル」を使用してください.
なお,フォントは,fixedを使用して下さい.本文中の半角英数字(参考文献番号含む)もMS明朝をお使い下さい.

\subsection{図表}

図表はなるべく本文に埋め込んでください.表キャプションは,上側中央に(表\ref{tab:font}),図キャプションは,下側中央に記載してください(図\ref{fig:logo}).
図表の前後には,できるだけ改行を入れてください.

\begin{table}[htb]
	\caption{フォントとポイント}
	\label{tab:font}
	\begin{tabularx}{\columnwidth}{llr}
		\hline
		\textbf{項目} & \textbf{フォント} & \textbf{ポイント} \\ \hline
		タイトル & MSゴシック & 14 \\
		サブタイトル & MSゴシック & 12 \\
		著者名 & MSゴシック & 12 \\
		所属 & MSゴシック & 11 \\
		メールアドレス & Arial & 11 \\
		研究概要 & MS明朝 & 9 \\
		各章の見出し & MSゴシック & 11 \\
		章番号 & Arial & 11 \\
		各節の見出し & MSゴシック & 10 \\
		節番号 & Arial & 10 \\
		本文 & MS明朝 & 10 \\
		図表キャプション & ゴシック & 10 \\
		参考文献(章題) & ゴシック & 10 \\
		参考文献(項目) & MS明朝 & 9 \\
		\hline
	\end{tabularx}
\end{table}


\begin{figure}[htbp]
\centering
\includegraphics[width=5cm]{jaeis-logo.pdf}
\caption{本学会の名称と英語名}
\label{fig:logo}
\end{figure}


また,投稿原稿にカラーの図表・写真等をお使いになることは可能ですが,学会誌は白黒印刷になります.

\subsection{句読点等}

原則として,「( )」,「英数」は半角を,「,」,「.」は全角を使用してください.

\subsection{参考文献}

参考文献は,以下の例を参考に記述してください\cite{jaeis-sample-article:en}.
(発行年)の前は,半角スペースです.

\begin{itemize}
\item 論文誌・雑誌の場合
	\cite{jaeis-sample-article:en}
	\cite{jaeis-sample-article:ja}
	\cite{jaeis-sample-article2:ja}
	著者名:``タイトル'',雑誌名,巻,号,ページ (発行年).
\item 書籍の場合
	\cite{jaeis-sample-book:ja}
		著者名,``書名'',参照ページ,発行所 (発行年).
\end{itemize}

また,本文中で参考文献
\cite{jaeis-sample-article:en}
\cite{jaeis-sample-article:ja}
\cite{jaeis-sample-article2:ja}
\cite{jaeis-sample-book:ja}
に関連する箇所には,このように参考文献番号を上付きで付与してください.

\section{原稿の提出について}

論文投稿は,EasyChairをご利用ください.EasyChair による論文投稿の手順は,学会のWebページから参照可能です.

\section{おわりに}

慣用的ではない用語については,本文または脚注 に説明を加えてください.

\section*{\hspace{9.5zw}謝辞}

謝辞を書く場合には,本文と参考文献の間に入れてください.

\vspace{-2zw}

\bibliographystyle{jaeis}

\bibliography{sample}

\end{document}  
